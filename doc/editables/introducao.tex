\chapter[O Projeto]{O Projeto}

\section{Introdução}
O projeto Robô para Crianças é vinculado a disciplina de “Projeto Integrador de Engenharia 2”, desenvolvido
no semestre 02/2015, busca contemplar o objetivo destacado, dentro desta, os conhecimentos e habilidades
técnicas adquiridas ao longo dos cursos de graduação em Engenharia Aeroespacial,  Engenharia Automotíva,
Engenharia de Energia, Engenharia Eletrônica e Engenharia de Software são utilizados para a solução de
problemas de engenharia. O projeto Robô para Criança está sendo  na disciplina de Projeto Integrador de
Engenharia 2. O grupo é composto por um estudante de Engenharia Aeroespacial, quatro de
Engenharia de Energia, três de Engenharia de Software, um de Engenharia Eletrônica. Nas próximas seções serão
apresentadas o contexto em que o projeto foi construído, os objetivos alcançados e a organização do projeto.


\section{Motivação}
Atualmente as tecnologias de Informação e Comunicação (TIC) tem influenciado na maneira em que a sociedade
interage, no lazer e no aprendizado. No que se diz respeito ao aprendizado, \citeonline{rauber:2003} afirma que
existem três habilidades para o processo de alfabetização que estudantes devem desenvolver: aprender a ler, aprender
a escrever e aprender a resolver problemas matemáticos – a partir da busca pelo melhoramento dos índices de
aprendizagem. Neste projeto decidiu-se pela aplicação de um objeto de aprendizagem, que é caracterizado de
acordo com \citeonline{tarouco:2004} que entende que esses objetos de aprendizagem “são recursos digitais usados para
apoiar o aprendizado” e ainda de acordo com (IEEE, 2002) um objeto de aprendizagem é um objeto qualquer, sendo
digital ou não, que seja referenciado ou utilizado para o suporte do aprendizado de alguma forma utilizando-se
a tecnologia.

A partir do ponto de vista de que a tecnologia deve exercer sua contribuição ao aprendizado em uma relevância
gradual, a proposta inicial deste projeto é a do incentivo ao desenvolvimento do raciocínio lógico durante o período
da infância, com a utilização de um robô (objeto de aprendizagem) que responde a comandos que envolvem uma
estrutura lógica e que também interage com o ambiente ao seu redor.

É importante citar que o autor \citeonline{weis:2001} menciona que “...existem crianças com baixo rendimento escolar
que, diante do computador, mostram-se mais participativas e interessadas”, por tanto planeja-se desenvolver um robô
atrativo (cores, luzes, formato e sons) para crianças operado através de comandos simples com grau de complexidade
que condizem com a faixa de idade que pretende-se trabalhar, tendo em vista que segundo \citeonline{piaget:1975},
é na infância (especificamente no período operatório formal) que as mesmas começam o desenvolvimento do pensamento
como o de um adulto, o que engloba o pensamento acerca do raciocínio logico e também de ideias abstratas. Uma vez
que houve ineficácia ao estímulo desses aspectos do pensamento, durante a infância, pode ser que se observe
dificuldades para compreensão e/ou resolução de problemas futuros mais complexos, tanto na matemática como em
outras áreas de conhecimento na vida universitária e/ou adulta \cite{rauber:2003}.Além de despertar a curiosidade
da criança para a resolução de problemas lógicos, também é considerado a evolução do interesse em robôtica e, por
consequencia, conhecimentos conexos.

Deseja-se auxiliar no processo de ensino-aprendizagem infantil através do exercícios e estímulo do raciocínio lógico,
utilizando-se uma ferramenta de largo e fácio acesso, onde, espera-se de resultado, posteriormente estudantes 
apresente melhores desempenhos em compreensão e contextualização, o que por sua vez facilitaria a interpretação 
de diferentes conteúdos inter-relacionandos, possivelmente enxergando problemas de maneira mais ampla, evitando 
memorizações desnecessárias e conhecimento superficial.

\section{Objetivos}
Dentro da área de robótica educacional existem várias aplicações que dependem dos seus objetivos. Godoy (1997), citado por Zilli (2004), apresenta algumas classificações de objetivos: gerais, psicomotores, cognitivos e afetivos. Tomando-se como base essa divisão, alguns objetivos definidos pelo autor foram adaptados para o contexto e aplicação deste projeto:
\begin{itemize}
	\item Reforçar a aprendizagem em programação por blocos;
	\item Introduzir conceitos de robótica;
	\item Proporcionar a curiosidade pela investigação levando ao desenvolvimento intelectual.
	\item Possibilitar resolução de problemas por meio de erros e acertos.
\end{itemize}

\section{Organização do Projeto}
O projeto Robô para Criânças concentra na visão de construção de um robô programável. Deste modo, este trabalho está dividido nas seguintes seções:
\begin{itemize}
	\item \textbf{Referências}: trata de aspectos relacionados a outros trabalhos propostos, retratando questões de robótica, arquitetura e lógica de programação;
	\item \textbf{Visão do Produto}: descrição técnica do produto alvo a ser produzido, tratando de questões estruturais, controle, processamento e operacionalização;
	\item \textbf{Implementação}: estado atual do projeto com relação a implementação da visão proposta do produto, trata de questões de ações
\end{itemize}
