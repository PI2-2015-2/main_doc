\chapter[O Projeto]{O Projeto}

\section{Introdução}
O projeto Robô para Crianças é vinculado a disciplina “Projeto Integrador de Engenharia 2” ofertada pela UnB - Faculdade Gama que tem como objetivo integrar os conhecimentos e as habilidades tecnicas adquiridas ao longo dos cursos de graduação na solucao de problemas, por meio do desenvolvimento de um tema real de projeto. Nas próximas seções serão apresentadas o contexto em que o projeto foi construido, os objetivos a serem alcançados e a organização do projeto.

\section{Motivação}
Atualmente as tecnologias de Informação e Comunicação (TIC) tem influenciado na maneira em que a sociedade interage, no lazer e no aprendizado. No que se diz respeito ao aprendizado, Rauber (2003) afirma que existem três habilidades para o processo de alfabetização que estudantes devem desenvolver: aprender a ler, aprender a escrever e aprender a resolver problemas matemáticos – a partir da busca pelo melhoramento dos índices de aprendizagem. Neste projeto decidiu-se pela aplicação de um objeto de aprendizagem, que é caracterizado de acordo com Tarouco (2004) que entende que esses objetos de aprendizagem “são recursos digitais usados para apoiar o aprendizado” e ainda de acordo com (IEEE, 2002) um objeto de aprendizagem é um objeto qualquer, sendo digital ou não, que seja referenciado ou utilizado para o suporte do aprendizado de alguma forma utilizando-se a tecnologia.

A partir do ponto de vista de que a tecnologia deve exercer sua contribuição ao aprendizado em uma relevância gradual, a proposta inicial deste projeto é a do incentivo ao desenvolvimento do raciocínio lógico durante o período da infância, com a utilização de um robô (objeto de aprendizagem) que responde a comandos que envolvem uma estrutura lógica e que também interage com o ambiente ao seu redor.

É importante citar que o autor Weis (2001) menciona que “...existem crianças com baixo rendimento escolar que, diante do computador, mostram-se mais participativas e interessadas”, por tanto planeja-se desenvolver um robô atrativo (cores, luzes, formato e sons) para crianças operado através de comandos simples com grau de complexidade que condizem com a faixa de idade que pretende-se trabalhar, tendo em vista que segundo Piaget (1975), é na infância (especificamente no período operatório formal) que as mesmas começam o desenvolvimento do pensamento como o de um adulto, o que engloba o pensamento acerca do raciocínio logico e também de ideias abstratas. Uma vez que houve ineficácia ao estímulo desses aspectos do pensamento, durante a infância, pode ser que se observe dificuldades para compreensão e/ou resolução de problemas futuros mais complexos, tanto na matemática como em outras áreas de conhecimento na vida universitária e/ou adulta \cite{rauber:2003}.

Outro aspecto que motiva o desenvolvimento deste robô além de despertar a curiosidade da criança para a resolução de problemas lógicos (praticar o raciocínio), é o de levar ao despertamento de interesse e consequentemente ao aumento de conhecimento.

Deseja-se auxiliar no processo de ensino-aprendizagem infantil através do exercícios e estímulo do raciocínio lógico, utilizando-se como ferramenta um robô descrito anteriormente, onde espera-se que posteriormente estudantes desempenhem melhores resultados, compreensão e maior contextualização, o que por sua vez facilitaria a interpretação de diferentes conteúdos inter-relacionando-os, possivelmente enxergando problemas de maneira mais ampla, evitando memorizações desnecessárias e conhecimento superficial.

\section{Objetivos}
Dentro da área de robótica educacional existem várias aplicações que dependem dos seus objetivos. Godoy (1997), citado por Zilli (2004), apresenta algumas classificações de objetivos: gerais, psicomotores, cognitivos e afetivos. Tomando-se como base essa divisão, alguns objetivos definidos pelo autor foram adaptados para o contexto e aplicação deste projeto:
\begin{itemize}
	\item Reforçar a aprendizagem em programação por blocos;
	\item Introduzir conceitos de robótica;
	\item Proporcionar a curiosidade pela investigação levando ao desenvolvimento intelectual.
	\item Possibilitar resolução de problemas por meio de erros e acertos.
\end{itemize}

\section{Organização do Projeto}
O projeto Robô para Criânças concentra na visão de construção de um robô programável. Deste modo, este trabalho está dividido nas seguintes seções:
\begin{itemize}
	\item \textbf{Referências}: trata de aspectos relacionados a outros trabalhos propostos, retratando questões de robótica, arquitetura e lógica de programação;
	\item \textbf{Visão do Produto}: descrição técnica do produto alvo a ser produzido, tratando de questões estruturais, controle, processamento e operacionalização;
	\item \textbf{Implementação}: estado atual do projeto com relação a implementação da visão proposta do produto, trata de questões de ações
\end{itemize}
